\documentclass[a4paper, 9pt]{article}
\linespread{1.33}
\usepackage{fullpage}
\usepackage[top=2cm, bottom=4.5cm, left=2.5cm, right=2.5cm]{geometry}
\usepackage{amsmath,amsthm,amsfonts,amssymb,amscd}
\usepackage{lastpage}
\usepackage{enumerate}
\usepackage{fancyhdr}
\usepackage{mathrsfs}
\usepackage{mathtools}
\usepackage{kotex}
\usepackage{xcolor}
\usepackage{graphicx}
\usepackage{listings}
\usepackage{physics}
\usepackage{tikz}
\usepackage{pgfplots}
\usepackage{enumitem}
\usetikzlibrary{decorations.text}
\pgfplotsset{compat = newest}
\setlength{\parindent}{0.0in}
\setlength{\parskip}{0.05in}
\newcommand\course{\textsc{마음당 전공이야기}}
\newcommand\hwnumber{1}
\newcommand\NetIDa{\textsc{@commuteneko}}
\newcommand\NetIDb{\textsc{Radical Halogenation}}

\pagestyle{fancyplain}
\headheight 40pt
\lhead{\NetIDa}
\lhead{\NetIDa\\\NetIDb}
\chead{\Large Topic \hwnumber}
\rhead{\course \\ \today}
\lfoot{}
\cfoot{마음당 전공이야기}
\rfoot{\small\thepage}
\headsep 1.5em

\newcommand{\pd}[2]{\frac{\partial{#1}}{\partial{#2}}}
\newcommand{\dirac}[3]{\left<{#1}\left|{#2}\right|{#3}\right>}

\begin{document}

\section{Introduction}
일반적으로, thermodynamics와 kinetics는 서로 관련되어 있지 않다. 하지만, 몇 가지 특수한 상황에서는 둘이 큰 연관을 가질 수 있다. 이번 글에서는 다음 concept들을 이해하는 것을 목표로 한다:
 \begin{itemize}
 	\item[-] Radical halogenation
 	\item[-] Selectivity of radical halogenation
 	\item[-] Hammond Postulate
 \end{itemize}
 \section{Radical Halogenation}
 \subsection{Bond-Dissociation Energy (BDE)}
 Radical halogenation 반응을 이해하기에 앞서, 몇 가지 결합들에 대한 BDE를 비교해 볼 필요가 있다.
 \begin{center}
 	\begin{tabular}{|c|c|c|c|c|}
 		\hline
 		type & methyl & $1^{\circ}$ & $2^{\circ}$ & $3^{\circ}$\\
 		\hline
 		formula & $\mathrm{CH_{3}-H}$ & $\mathrm{RCH_{2}-H}$ & $\mathrm{R_{2}CH-H}$ & $\mathrm{R_{3}C-H}$ \\
 		\hline
 		$DH^{\circ}\,(\mathrm{kcal/mol})$ & 105 & 101 & 98.5 & 96.5 \\
 		\hline
 	\end{tabular}
 \end{center}
H 원자가 떨어져 나간 이후의 carbocation이 안정할수록 BDE가 작아진다. 중심의 C 원자를 기준으로 substitution이 늘어날수록 hyperconjugation에 의한 carbocation의 stabilization이 커지므로, 위와 같은 경향성을 이해할 수 있다. 그리고 결합의 세기로부터, 결합 길이는 반대의 경향성을 가질 것으로 예측할 수 있다.

\subsection{Mechanism of Radical Halogenation}
학부 유기화학에서 가장 처음 배우는 반응의 메커니즘을 알아보자. 우선, 가장 간단한 alkane인 methane의 chlorination 반응을 살펴보자.
\[ \mathrm{ CH_{3}-H\;+\;Cl-Cl\;\xrightarrow[\Delta\;\text{or}\;h\nu]{}\;CH_{3}-Cl\;+\;H-Cl } \]
위의 반응은 reactant들에 UV light를 쬐어 주거나 300$^{\circ}$C 이상으로 가열했을 때 관찰된다. 위와 같은 halogenation 반응의 mechanism은 \textbf{radical chain mechanism}이라고 알려져 있다. 여기서 radical이란 홀전자를 가지는 화학종을 의미한다. 위의 반응은 크게 세 가지 단계로 나눠진다.
\begin{itemize}
	\item[(1)] \textbf{Initiation}: UV나 가열에 의해서 Cl 기체 분자가 homolytic cleavage (single bond의 전자를 골고루 나눠 갖는)를 거친다.
	\[ \mathrm{ Cl-Cl\;\xrightarrow[\Delta\;\text{or}\;h\nu]{}2Cl\cdot\qquad}\Delta{H}^{\circ}=+58\text{kcal/mol} \]
	중요한 점은, 모든 reactant의 chlorine gas 분자가 위와 같은 initiation 과정을 거칠 필요는 없다는 점이다. 사실 위의 반응은 꽤나 endothermic하므로 썩 자주 일어나지는 않지만, 미량의 Cl radical이 생성되기만 하더라도 이후의 반응들이 연쇄적으로 일어나므로 문제는 없다.
	\item[(2)] \textbf{Propagation}: Initiation step에서 만들어진 Cl radical이 methane을 공격한다.
	\[ \mathrm{ Cl\cdot \;+\; H-CH_{3}\;\rightarrow\;Cl-H\;+\; \cdot{C}H_{3}\qquad}\Delta{H}^{\circ}=+2\text{kcal/mol} \]
	이 과정에서 생성된 methyl radical($\mathrm{CH_{3}\cdot}$)이 또다른 $\mathrm{Cl_{2}}$를 공격해서 Cl radical을 재생성한다. 이 과정이 계속해서 반복되며 연쇄 반응을 일으킨다.
	\[ \mathrm{ H_{3}C\cdot\;+\;Cl-Cl\;\rightarrow\;H_{3}C-Cl\;+\;Cl\cdot\qquad}\Delta{H}^{\circ}=-27\text{kcal/mol} \]
	즉 앞에서 언급한대로, initiation 단계에서 미량의 Cl radical이 생성되더라도, 계속된 연쇄반응으로 반응은 이어질 수 있다. Propagation의 첫 번째 단계는 slightly endothermic하지만, 두 번째 단계가 exothermic하므로 반응 전체의 driving force가 된다.
	\item[(3)] \textbf{Termination}: 반응이 종결될 때는, 반응에 참여하는 radical들이 서로 조합되어 생성물을 만든다.
	\[ \mathrm{Cl\cdot\;+\;Cl\cdot\;\rightarrow\;Cl-Cl} \]
	\[ \mathrm{Cl\cdot\;+\;CH_{3}\cdot\;\rightarrow\;Cl-CH_{3}} \]
	\[ \mathrm{CH_{3}\cdot\;+\;CH_{3}\cdot\;\rightarrow\;H_{3}C-CH_{3}} \]
\end{itemize}
\subsection{Other Radical Halogenation of Methane}
Cl을 제외하고도, 17족에는 다른 halogen들이 있다. 다른 종류의 halogen에 대해 각 step의 energetics를 관찰해 보면 다음과 같다. 단위는 모두 kcal/mol이다.
\begin{center}
	\begin{tabular}{|c|c|c|c|c|}
		\hline
		Reaction & F & Cl & Br & I \\
		\hline
		$\mathrm{X\cdot\;+\;CH_{4}\;\rightarrow\;\cdot{C}H_{3}\;+\;HX}$ & -31 & +2 & +18 & +34 \\
		$\mathrm{\cdot{C}H_{3}\;+\;X-X\;\rightarrow\;X-CH_{3}\;+\;X\cdot}$ & -72 & -27 & -24 & -21 \\
		\hline
		\textcolor{red}{$\mathrm{CH_{4}\;+\;X-X\;\rightarrow\;X-CH_{3}\;+\;H-X}$} & -103 & -25 & -6 & +13 \\
		\hline
	\end{tabular}
\end{center}
생성되는 $\mathrm{H-X}$ 결합의 세기에 따라서 반응 전체가 exothermic한지 endothermic한지 결정되는 모습을 관찰할 수 있다. 위의 표로부터, radical iodination는 일반적으로 일어나지 않음을 예측할 수 있다. 그리고, Cl이 아닌 다른 halogen에 대해서도 앞 절에서와 동일한 mechanism을 가진다는 사실을 굳이 그려보지 않아도 알 수 있다.

\subsection{Selectivity of Radical Halogenation}
Methane 또는 ethane과 같이 간단한 alkane 말고, 좀 더 복잡한 alkane의 radical halogenation을 생각해 보자. 그 다음으로 복잡한 propane($\mathrm{C_{3}H_{8}}$)의 chlorination과 bromination을 살펴보자. 이 결과는 매우 흥미롭다.
\begin{figure*}[htbp]
\centering
\includegraphics[width=0.53\linewidth]{01_1.png}	
\end{figure*}

Propane은 6개의 primary hydrogen과 2개의 secondary hydrogen을 가지고 있으므로, 2가지의 radical halogenation product가 생성된다. 그래서, 이론적으로는 primary H가 치환된 product와 secondary H가 치환된 product의 비율이 3:1, 즉 75\%와 25\%를 차지해야 할 것처럼 여겨진다. 그러나 chlorine에서는 그 둘의 비율이 거의 비슷하며, 심지어 bromine에서는 거의 secondary H가 치환된 product가 대부분을 차지한다. 이를 어떻게 설명할 수 있을 것인가?
\subsection{Hammond's Postulate}
\begin{figure*}[htbp]
\centering
\includegraphics[width=0.7\linewidth]{01_2.png}	
\end{figure*}
어떤 반응에 대해서, reactant와 transition state 사이의 형태가 비슷하다면, 그 에너지 차이가 크지 않을 것이므로 reorganization energy도 작을 것이라고 추측할 수 있다.
\begin{itemize}
	\item[-] Exothermic한 반응이라면, 상대적으로 activation barrier ($E_{a}$)도 작고, transition state의 형태도 reactant와 유사하다 (Early TS, reactant-like).
	\item[-] Endothermic한 반응이라면, 상대적으로 activation barrier가 높고 따라서 transition state의 형태는 product에 더 가깝다 (Late TS, product-like).
\end{itemize}
이를 \textbf{Hammond's postulate}라고 부른다. Hammond's postulate를 바탕으로 앞 절의 결과를 설명해 보자.
\begin{itemize}
	\item[-] Halogen radical이 어디를 공격하는지 역시 중요하다. Primary radical보다 secondary radical이 더 안정하다. 이유는 역시 hyperconjugation.
	\item[-] 그러나, chlorination reaction의 driving force (전체 $\Delta{H}^{\circ}$의 크기)가 더 크다. 즉, driving force 자체가 큰 chlorination reaction에 대해서는 secondary radical이 조금 더 안정하다는 사실이 ``뭉개진다".
	\item[-] 반면에 bromination reaction은 그 driving force 자체가 약하므로, radical의 stability가 반응물의 분포에 큰 영향을 준다.
	\item[-] 상대적으로, radical chlorination의 transition state는 reactant-like하다. 반대로, radical bromination의 transition state는 product-like하다.
	\begin{figure*}[htbp]
		\centering
		\includegraphics[width=\linewidth]{01_3.png}	
	\end{figure*}
	\item[-] 따라서, radical chlorination에서는 activation barrier 자체가 낮으며, 두 가지 product를 향한 pathway 모두를 쉽게 지날 수 있다.
	\item[-] 반면, late TS를 가지는 radical bromination은 activation energy의 차이가 product distribution에 큰 영향을 준다.
\end{itemize}
Radical fluorination의 경우에는, driving force의 크기가 chlorination에서보다 더 크다. 따라서, statistical ratio (3:1)에 훨씬 가까운 숫자를 얻게 된다. 정리하자면, transition state가 early인지 late인지 그 정도에 따라서, statistical ratio에서 product distribution이 얼마나 벗어나는지를 알 수 있다.
\section{Conclusion}
Radical chain mechanism은 앞서 언급했듯이 유기화학에서 가장 처음 배우는 반응 메커니즘이다. 즉 그만큼 중요하면서도 이해하기 쉽다고 할 수 있을 것이다. 간단한 예시로부터 physical organic chemistry의 중요한 개념 중 하나인 Hammond's postulate에 대해 얕은 수준의 이해를 해볼 수 있을 것으로 기대한다.
\section{References}
\begin{enumerate}
	\item Vollhardt, P.; Schore, N. \textit{Organic Chemistry}, 8th ed.; W. H. Freeman, 2018.
	\item Carey, F. A.; Sunderberg, R. J. \textit{Advanced Organic Chemistry Part A: Structure and Mechanisms}, 5th ed.; Springer, 2007.
\end{enumerate}
\end{document}



















































